\documentclass[11pt,a4paper]{article}

\usepackage[T1]{fontenc}
\usepackage[utf8]{inputenc}
\usepackage[ngerman]{babel}
\usepackage[a4paper,margin=1.6cm]{geometry}
\usepackage{lmodern}
\usepackage{graphicx}
\usepackage{tabularx}
\usepackage{enumitem}
\usepackage[hidelinks]{hyperref}
\usepackage{xcolor}

\pagestyle{empty}
\setlength{\parindent}{0pt}
\setlist[itemize]{leftmargin=*,itemsep=1pt,topsep=1pt,parsep=0pt,partopsep=0pt}

\newcommand{\sectionspacebefore}{0.85em}
\newcommand{\sectionspacetoline}{-0.12em}
\newcommand{\sectionspacetocontent}{0.48em}
\newcommand{\roleblockgap}{0.52em}

\newcommand{\cvsection}[1]{
  \par\vspace{\sectionspacebefore}
  {\large\bfseries\color{black}#1}\par
  \vspace{\sectionspacetoline}
  \noindent\rule{\textwidth}{0.45pt}\par
  \vspace{\sectionspacetocontent}
}

\newcommand{\cvname}{Matthias Hutter}
\newcommand{\cvheadline}{Senior Software Architect | Infrastruktur und Finanzdaten-Systeme}
\newcommand{\cvaddress}{Bruno-Marek-Allee 10/64, 1020 Wien}
\newcommand{\cvphone}{+43 650 8223546}
\newcommand{\cvemail}{hutter.matthias@gmail.com}
\newcommand{\cvgithub}{github.com/ScepticMatt}
\newcommand{\cvlinkedin}{linkedin.com/in/matthias-hutter-7837349a}

\newcommand{\birthdate}{27.08.1989}
\newcommand{\nationality}{Österreich}

\newcommand{\role}[4]{
  \textbf{#1} \hfill #2\\
  \textit{#3} \hfill \textit{#4}
  \vspace{0.15em}
}

\begin{document}

\begin{minipage}[t]{0.73\textwidth}
\vspace{0pt}
{\LARGE\textbf{\cvname}}\\[0.2em]
{\large\cvheadline}\\[0.8em]
\begin{tabular}{@{}ll@{}}
Adresse: & \cvaddress \\
Telefon: & \cvphone \\
E-Mail: & \href{mailto:\cvemail}{\cvemail} \\
GitHub: & \href{https://\cvgithub}{\cvgithub} \\
LinkedIn: & \href{https://\cvlinkedin}{\cvlinkedin} \\
\end{tabular}
\end{minipage}
\hfill
\begin{minipage}[t]{0.23\textwidth}
\vspace{0pt}
\raggedleft
\IfFileExists{crop.jpg}{\includegraphics[width=0.80\linewidth]{crop.jpg}}{\fbox{\parbox[c][3.8cm][c]{0.8\linewidth}{\centering Foto}}}
\end{minipage}

\vspace{-0.1em}
\cvsection{Persönliche Daten}
\begin{tabularx}{\textwidth}{@{}lX@{}}
Geburtsdatum: & \birthdate \\
Staatsangehörigkeit: & \nationality \\
Wohnort: & Wien, Österreich \\
\end{tabularx}

\cvsection{Kurzprofil}
\begin{itemize}
  \item Senior Software Architect und Infrastrukturverantwortlicher mit 8+ Jahren Erfahrung in FinTech- und Digital-Asset-Umgebungen.
  \item End-to-End-Verantwortung von Architektur und Umsetzung bis zu CI/CD, Monitoring, Incident Response und stabilem Produktionsbetrieb.
  \item Fundierte Erfahrung in auditierbaren Finanzdaten-Pipelines und Reporting (NAV, Balance, PnL) für regulierte Handelsprozesse.
  \item Hands-on-Führungserfahrung in Recruiting, Mentoring und teamübergreifender Delivery in verteilten und internationalen Teams.
\end{itemize}

\cvsection{Berufserfahrung}
\role{Director of Infrastructure}{Jan 2022 -- heute}{Dark Forest Limited (inkl. vorheriger Rolle bei Dark Forest GmbH)}{Wien / Remote}
\begin{itemize}
  \item Führung von Infrastruktur-Architektur und Automatisierung vom Unternehmensaufbau bis zum heutigen regulierten Electronic-Trading-Betrieb.
  \item Aufbau und Betrieb einer Validator-Landschaft über \textasciitilde20 PoS-Netzwerke (\textasciitilde60 Validatoren) auf Cloud- und Bare-Metal-Infrastruktur.
  \item Konzeption und Umsetzung des Accounting Data Managers (ADM) als auditierbare Golden Source aus 10 Exchange-Feeds und internen Metadaten.
  \item Etablierung von automatisiertem NAV-, Balance- und PnL-Reporting via API, Dashboard, E-Mail, Slack und Excel.
  \item Entwicklung interner Services für Blockchain- und Marktdaten zur Unterstützung von Trading und Reporting.
\end{itemize}
\textbf{Technologien:} Node.js, TypeScript, Python, Java, OpenAPI, PostgreSQL, kdb+, AWS, OVH, Terraform, Ansible, Kafka, RabbitMQ, Prometheus, React, Keycloak

\vspace{\roleblockgap}
\role{Lead Technical Architect / Backend Team Lead}{Sep 2017 -- Dez 2022}{Kivu Technologies}{Wien}
\begin{itemize}
  \item Entwicklung vom Backend Developer zum Team Lead und später zum Lead Technical Architect; in der Übergangsphase 2022 während des Wechsels zu Dark Forest geringfügig weitergeführt.
  \item Leitung einer umfassenden Modernisierung von Legacy-Systemen in Richtung bi-temporaler Time-Series-Graph-Architektur.
  \item Entwurf von Datenmodellen, Graph-Komponenten und automatisierten Datenpipelines für Enterprise-Integrationen.
  \item Verantwortung für Backend-Roadmap, Recruiting, Mentoring und Delivery in enger Abstimmung mit Produkt, QA und Kunden.
\end{itemize}
\textbf{Technologien:} Java, Vert.x, Spring, PostgreSQL, MySQL, Redis, RocksDB, Lucene, Jenkins

\vspace{\roleblockgap}
\role{University Tutor (Computer Science)}{Jan 2016 -- Mär 2017}{URACCAN}{Nicaragua}
\begin{itemize}
  \item Unterricht in Algorithmen, Komplexität und Programmierung in C.
  \item Betreuung von Lehramtspraktika und Unterstützung der Studienprogramme.
\end{itemize}

\cvsection{Technische Kenntnisse}
\textbf{Kernkompetenzen:} Plattform-Architektur, Backend-Engineering, Finanzdaten-Systeme, Infrastruktur-Automatisierung, Teamführung\\
\textbf{Programmiersprachen:} TypeScript, Java, Python, C/C++, JavaScript, SQL\\
\textbf{Backend und APIs:} Node.js, Express, Vert.x, Spring, REST, OpenAPI, RabbitMQ, Kafka\\
\textbf{Daten und Storage:} PostgreSQL, MySQL, Redis, RocksDB, Apache Lucene, kdb+\\
\textbf{Infrastruktur und Betrieb:} AWS, Linux, Terraform, Ansible, Docker, CI/CD, GitHub Actions\\
\textbf{Monitoring und Identity:} Prometheus, Alertmanager, Grafana, OIDC, RBAC, Keycloak

\cvsection{Ausbildung}
\textbf{Studium (ohne Abschluss): Electronics Engineering} \hfill 2011 -- 2014\\
Swiss Federal Institute of Technology (ETH Zurich)

\vspace{0.15em}
\textbf{Studium (ohne Abschluss): Physik} \hfill 2008 -- 2011\\
Swiss Federal Institute of Technology (ETH Zurich)

\vspace{0.15em}
\textbf{Elektrotechnik-Lehre + Matura} \hfill 2003 -- 2008\\
Werkschulheim Felbertal

\end{document}
